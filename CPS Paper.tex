%%%%%%%%%%%%%%%%%%%%%%%%%%%%%%%%%%%%%%%%%%%%%%%%%%%%%%%%%%%%%%%%%%%%%%%%%%%%%%%%
%2345678901234567890123456789012345678901234567890123456789012345678901234567890
%        1         2         3         4         5         6         7         8

\documentclass[letterpaper, 10 pt, conference]{ieeeconf}  % Comment this line out
                                                          % if you need a4paper
%\documentclass[a4paper, 10pt, conference]{ieeeconf}      % Use this line for a4
                                                          % paper

\IEEEoverridecommandlockouts                              % This command is only
                                                          % needed if you want to
                                                          % use the \thanks command
\overrideIEEEmargins
% See the \addtolength command later in the file to balance the column lengths
% on the last page of the document



% The following packages can be found on http:\\www.ctan.org
%\usepackage{graphics} % for pdf, bitmapped graphics files
%\usepackage{epsfig} % for postscript graphics files
%\usepackage{mathptmx} % assumes new font selection scheme installed
%\usepackage{times} % assumes new font selection scheme installed
%\usepackage{amsmath} % assumes amsmath package installed
%\usepackage{amssymb}  % assumes amsmath package installed
\usepackage{comment}
\title{\LARGE \bf
Interdependent Cyber-Physical Systems: On the Effectiveness of Standard Centrality Metrics
}

% \author{ \parbox{3 in}{\centering Huibert Kwakernaak*
%         \thanks{*Use the $\backslash$thanks command to put information here}\\
%         Faculty of Electrical Engineering, Mathematics and Computer Science\\
%         University of Twente\\
%         7500 AE Enschede, The Netherlands\\
%         {\tt\small h.kwakernaak@autsubmit.com}}
%         \hspace*{ 0.5 in}
%         \parbox{3 in}{ \centering Pradeep Misra**
%         \thanks{**The footnote marks may be inserted manually}\\
%        Department of Electrical Engineering \\
%         Wright State University\\
%         Dayton, OH 45435, USA\\
%         {\tt\small pmisra@cs.wright.edu}}
% }

 \author{Nathaniel Hudson, Matthew Turner, Asare Nkansah, Hana Khamfroush} % <-this % stops a space
% \thanks{*This work was not supported by any organization}% <-this % stops a space
% \thanks{$^{1}$H. Khamfroush is with the Department of Computer Science,
%        University of Kentucky, USA}
%         {\tt\small h.kwakernaak at papercept.net}}%
% \thanks{$^{2}$P. Misra is with the Department of Electrical Engineering, Wright State University,
%        Dayton, OH 45435, USA
%        {\tt\small p.misra at ieee.org}}%
% }


\begin{document}



\maketitle
\thispagestyle{empty}
\pagestyle{empty}

%%%%%%%%%%%%%%%%%%%%%%%%%%%%%%%%%%%%%%%%%%%%%%%%%%%%%%%%%%%%%%%%%%%%%%%%%%%%%%%%

\begin{abstract}
This paper investigates the effectiveness of standard centrality metrics for cyber-physical systems (CPS) in preventing catastrophic failure. To show the need of designing customized centrality metrics for interdependent cyber-physical systems, we compare the performance of these metrics in a CPS under two different scenarios: i) the nodes with highest centrality of networks composing an interdependent CPS are selected separately, ii) the nodes with highest centrality of the entire interdependent CPS represented as one single network are calculated. To investigate the resiliency of the interdependent CPS, a threshold-based failure propagation model is used to simulate the evolution of failure propagation over time. The nodes with highest centrality are chosen and are assumed to be resistant w.r.t failure. Extensive simulation is conducted to compare the usefulness of standard metrics to stop or slow down the failure propagation in a CPS. Finally, useful guidelines on designing new metrics are presented.
%This paper investigates the effectiveness of standard centrality metrics for interdependent networks in preventing catastrophic failure. To show the need of designing customized centrality metrics for interdependent networks, we compare the performance of these metrics under two different scenarios: i) the nodes with highest centrality of networks composing an interdependent network are selected separately, ii) the nodes with highest centrality of the entire interdependent network represented as one single network are calculated. To investigate the resiliency of the interdependent networks, a threshold-based failure propagation model is used to simulate the evolution of failure propagation over time. The nodes with highest centrality are chosen and are assumed to be resistant w.r.t failure. Extensive simulation is  conducted to compare the usefulness of these metrics to stop or slow down the failure propagation. The results of this study show that ...

\end{abstract}

%%%%%%%%%%%%%%%%%%%%%%%%%%%%%%%%%%%%%%%%%%%%%%%%%%%%%%%%%%%%%%%%%%%%%%%%%%%%%%%%
\section{INTRODUCTION}
Cyber-physical systems (CPS) are systems that are built from, and depend on the seamless integration of computation and physical components. These systems exploit sensing, computation, communication, and actuation to infuse engineered systems with intelligence and capabilities~\cite{c1}. The interdependency between cyber component and physical component creates specific characteristics for these systems that needs to be investigated.
In the recent years, the research on CPS mainly aimed at improving the ability to understand the interdependency between cyber and physical worlds and to design resilient and robust CPS.
%Today's critical infrastructures, such as power grids, water networks, food and agriculture networks, emergency networks, and etc, are forming interdependent networks, where the functionality or performance efficiency of one network depends on the other networks or is affected by other networks.
%Investigating the interdependency between these systems and its impact on the resiliency of the cyber-physical systems would be of great importance.
There is a considerable amount of literature addressing the problem of modeling and analyzing the impact of interdependency between different networks including cyberphysical systems, and how can we control the phenomena propagation in these networks. However, quantifying the importance of the nodes in such systems remain less investigated.

Centrality metrics in graph theory and sociology are used to quantify the importance of nodes, however, the state-of-the-art metrics are initially designed for a single undirected network, while their effectiveness for an interdependent cyber-physical system has not been studied extensively. In this paper, we focus on studying the effectiveness of the state-of-the-art centrality metrics in the context of interdependent networks. Our previous work showed that the selection of the important nodes of an interdependent network will have a great impact on the speed of phenomena propagation, and therefore can control the resiliency of the network. For a given interdependent network consisting of two networks A and B, and a specific level of coupling between the two networks, the following questions will be answered in this paper:
\begin{itemize}
\item \textit{How should we quantify the importance of the nodes in an interdependent network?}
\item \textit{If we consider an interdependent network as a single graph, how precise the state-of-the-art centrality metrics could perform to measure the node's importance ?}
\item \textit{What are  possible modifications we need to apply to the standard centrality metrics to adapt those metrics for interdependent networks?}
\item \textit{Which of the standard centrality metrics should be modified, if any?}
\end{itemize}
To the best of our knowledge this is the first work extensively studying the usefulness of standard centrality metrics for interdependent cyber-physical systems. We believe that the results presented in this paper will contribute to the design of meaningful centrality metrics for interdependent networks. This will in turn leads to determining the nodes that need to be protected against failure/attack to have more resilient and robust interdependent cyber-physical systems.
\section{Related Work}
There are several publications that discuss the various components involved in preventing the catastrophic failure of an interdependent cyber-physical system. Most of these papers focus on the modeling and construction for such an interdependent network; a common, widely used example is an electric cyber-physical infrastructure (power grid model) [2]. This model is based on an event that occurred in 2003, where a combination of power-component failures and human error contributed to the cascading failures that ultimately led to one of the largest blackouts in US history. The paper effectively communicates the factors that led to the catastrophic failure and provides recommendations for protocols to avoid human failure. Although, it never uses the data to examine the specific faults within the CPS.

Based on the power grid model, other papers do explore possible methods for preventing failure while producing a robust interdependent CPS [3], [4]. [3] proposes their own designs for the interdependent networks, focusing on optimizing different topological properties such as degree correlation, network modularity, node clustering coefficients, etc. [4] constructs a framework to predict the CPS’s resilience to cascading failures. Both of these papers introduce effective methods to approach interdependent systems, but they fail to characterize the properties of the individual nodes in the system and enumerate their importance.

In [5], the authors discuss the spread of an agent through an epidemic model propagation. While the epidemic model used in the paper works effectively, we decided to use the threshold propagation model. This model can be manipulated to construct other types of models, like the epidemic, etc.

In order to understand the properties of nodes across the independent network, centrality metrics are the best method of measurement. [6] identifies different centrality metrics and uses them to alter the flow of information across a social network. The paper accurately depicts the various metrics (betweenness, closeness, and path degree) and their characteristics, but their results are only calculated over a single layer network. Unlike a singular layer network, interdependent network’s nodes carry different importance according to their placement within their layer of the network.

\section{Problem Statement and Network Model}
We consider an interdependent cyber-physical system consisting of two networks, namely, the physical resource ($P$) and the computational resource network ($C$). Both networks, $P$ and $C$, are represented by two undirected graphs, $G_{P}=(V_{P},E_{P})$ and $G_{C}=(V_{C},E_{C})$. The two networks are interconnected by means of directed links. We refer to the edges that connect the nodes within the individual networks as \textit{intra-link} and those that connect the nodes from different networks as \textit{inter-link}. We use directed edges for inter-links to capture different models of inter-dependency between cyber and physical networks. Without loss of generality, it is assumed that an initial set $F_0$ of nodes in network \textit{P} are failed initially. Failure can be propagated among the nodes inside network \textit{P} and from nodes of \textit{P} to the nodes in network \textit{C} following a threshold-based propagation model as discussed in the next Section.

\subsection{Random Networks}
In the study of network science, random networks are commonly created to test novel algorithms, failure models, phenomena propagation, and centrality metrics. For this work, we take advantage of the following random network topologies: i) Erdős–Rényi (ER) network, ii) small-world (SW) network, and iii) scale-free (SF) network.

ER networks are generated by creating edges between $N$ nodes with some probability $P$. The benefit of the ER network model is in that its average degree, \textbf{\textless $k$\textgreater}, is approximately equal to $N \cdot P$ \textbf{[ref]}.

SW networks are networks in which most nodes are not connected to most other nodes in a network. However, the key property of a SW network is that to reach one node from another node, only a small number of hops are necessary. To be more formal, the small-world property is that the distance, $L$, between two randomly chosen nodes grows at a logarithmic rate with respect to $N$, such that $L \propto \log{N}$ [ref]. For this work, we use the the \textit{Watts-Strogatz} (WS) model for random SW network generation. The WS model creates a random SW network by creating a ring-lattice network of $N$ nodes and $K$ neighbors. Upon doing so, each node will rewire its edges with a probability of $\beta$ \textbf{[ref]}.

SF networks are networks with degree distribution asymptotically following a power law. Formally speaking, some fraction of nodes $P(k)$ in the network having $k$ edges to other nodes for larger values of $k$, $P(k) \sim k-\gamma$. Typically, the value $\gamma$ is within the range of $2 < \gamma < 3$. SF networks are commonly marked by having hubs with high centrality or a large number of edges compared to the majority of the network topology \textbf{[ref]}. For this work, we employ \textit{Barab\'{a}si-Albert} (BA) model for random SF network generation. This model follows the thinking of the Matthew Effect, as in "the rich get richer" \textbf{[ref]}. The model creates a connected network with initially $M_{0}$ nodes. A new node is added to the network until there is a total of $N$ nodes in the network. When a new node is added, it is connected to $m \le M_{0}$ pre-existing nodes with a probability proportional to the number of edges the pre-existing nodes already have. A new node will connect to node i with the probability $p_{i}=k_{i}/\sum_{j}{k_{j}}$, $k_{i}$ is the degree of node $i$.





\subsection{Interdependent Network (IDN)}
For this study, we are investigate failure propagation in interdependent networks. We can model a cyber-physical system using an interdependent network such that network C and network P are connected through dependency links. Intra-links in either network are links that do not cross from one network to another. Interlinks connect one node from either network to some node in the other network. In our model, intra-links are bidirectional; inter-links are unidirectional. For this work, we use generate random networks under \textbf{homogeneous} and \textbf{heterogeneous} topologies. We also investigate using IEEE300 data modeling the US power-grid to provide a real-world example to compare against randomly generated topologies.

The topology that results from the BA model differs greatly from the ER model. Homogeneous topologies are IDN network topologies in which both networks, C and P, are generated using the same random network model. Homogeneous IDN topologies for this study include ER-ER, SW-SW, and SF-SF.  Heterogeneous topologies are IDN network topologies in which both networks, C and P, are generated using different random network models. Heterogeneous IDN topologies for this study include ER-SW, ER-SF, and SW-SF. We do not consider ER-SW to differ from SW-ER topologies. Note that from this point forward, SW networks will be generated using the WS model; SF networks will be generated using the BA model.





\subsection{Simulations}
Upon creating random and real-world IDN topologies, we run several simulations in which an initial set of nodes are failed and phenomena propagation takes place. Simulations track and record the data of the phenomena propagation over a series of T time-steps. To account for the randomness of the network generation process, each simulation is run S times. Upon completion, the data collected during the simulation phase is averaged to find the average failure through phenomena propagation. Simulations are run against various different IDN topologies (as described earlier), network sizes, parameters for network generation, and phenomena parameters.





\section{Propagation model}
We adopt a probabilistic threshold-based failure propagation model that was proposed in ~\cite{c2}, as it can cover many other propagation models from epidemic models to other threshold models presented by []. Without loss of generality, we assume that nodes belonging to the same network have peer roles. For this reason we use undirected links to represent connections of nodes of the same network and model them through a symmetric adjacency matrix. We denote with $A_{CC}\in \{0,1\}^{n_{C}\times n_{C}}$ and $A_{PP}\in \{0,1\}^{n_{P}\times n_{P}}$ the symmetric adjacency matrices representing the intra-links of networks $C$ and $P$, respectively. The two networks are interconnected by means of directed inter-links, according to the inter-connection matrices $A_{CP}\in \{0,1\}^{n_{C}\times n_{P}}$ and $A_{PC}\in \{0,1\}^{n_{P}\times n_{C}}$. We use directed edges for inter-links to capture different models of inter- dependency between heterogeneous networks. Two nodes connected through an intra-link are intra-neighbors. The set of intra-neighbors of node $i\in C$ is denoted by $U_{C}^{intra}(i)=\{j\in {C}\mid (i,j)\in E_C\}$, while the value of $\mid U_{C}^{intra}(i) \mid$ is the intra-degree of node $i\in C$. Similar notations are introduced for network P. Given a node $i\in C$, we define the set of inter-parents of i, as $U_{C}^{inter}(i)=\{j\in {P}\mid A_{PC}(j,i)=1\}$. We refer to $\mid U_{C}^{inter}(i) \mid$ as the inter-degree of node $i\in C$. Similar notation is used for the interconnecting edges from network C to network P.

According to this model a node may fail only if the number of failed neighbors exceeds a given threshold.  Much of the existing literature also works under a threshold model, but generally works under a deterministic model such that failure is guranteed after passing the threshold.  For the purpose of this paper, we consider a probabilistic model based on a specified $P_{max}$ for each set of connectivity i.e. $P_{maxCC}$, $P_{maxCP}$, etc.  We make no assumption about the relationship between these values.

\subsection{$P_{AB}(x)$ Formulation}
$P_{maxAB}$ is obtained for a certain node in network A only when all of its neighbors in network B are currently afflicted with the propagating phenomena.  When the threshold of failed neighbors, $K_{AB}$, has been reached, the probability of failure is $P_{minAB}$.  It is assumed that there is linear growth from $P_{min}$ to $P_{max}$ as the proportion of failed neighbors grows from $K_{AB}$ to 1.  We express $P_{AB}(x)$ as the probability of failure with respect to x, the proportion of neighbors that have failed where:

{\tiny
 $
\begin{array}{cc}
	P_{AB}(x) = \{ &
    	\begin{array}{cc}
        0 & x < K_{AB} \\
        \frac{P_{max}-P_{min}}{1-K_{AB}}(x - K_{AB}) - P_{minAB} & x \geq K_{AB}
      	\end{array}
\end{array}
$}
\newline

Note that due to the linearity of $P_{AB}(x)$: \[
\frac{P_{maxAB}}{1} = \frac{P_{minAB}}{K_{AB}}
\]
\[
P_{maxAB}\times K_{AB} = P_{minAB}
\]

Thus, with substitution and simplification we find:

$P_{AB}(x)$ =
$
\begin{array}{cc}
	\{ &
    	\begin{array}{cc}
        0 & x < K_{AB} \\
        P_{maxAB} \times x & x \geq K_{AB}
      	\end{array}
\end{array}
$
%To study the effectiveness of different centrality metrics for interdependent networks, we assume that $K$ nodes of the network with highest centrality are selected to be reinforced nodes, meaning that these nodes will not fail during the whole process of phenomena propagation.
%We study the time evolution of failure propagation given different initial failure set, and different sets of reinforced nodes selected based on different centrality metrics.

\section{Centrality Metrics Effectiveness Analysis} In the following, we will examine the efficiency of the state-of-the-art centrality metrics in quantifying the importance of the nodes for cyber-physical interdependent networks. An initial set of failure in the cyber network is chosen randomly, and the evolution of failure over time is investigated for the interdependent CPS. To evaluate the usefulness of the state-of-the-art centrality metrics, we select $K$ nodes with highest centrality and prevent them from failure by reinforcing those nodes, meaning that these nodes will not fail during the whole process of phenomena propagation. We study the time evolution of failure propagation given different initial failure set, and different sets of reinforced nodes selected based on different centrality metrics.

For each centrality measure, we will define two versions: a) local centrality metric, which is defined according to standard centrality metrics for network \textit{C} and \textit{P}, separately, and b)global centrality metric, which considers the whole interdependent network as a single graph, and applies standard centrality metrics to this graph.
\subsection{Usefulness of Degree centrality} Degree-centrality is the simplest centrality measure that counts how many neighbors a node has. We calculate global degree centrality ($D_g$) and local degree centrality ($D_l$) of the interdependent networks.
\subsection{Usefulness of Betweenness centrality} Betweenness centrality of a vertex measures the extent to which a vertex lies on paths between other vertices. Vertices with high betweenness may have considerable influence within a network by virtue of their control over information passing between others. We calculate global betweenness centrality ($B_g$) and local betweenness centrality ($B_l$) of the interdependent network.
\subsection{Usefulness of Eigenvector centrality} Degree centrality assigns one centrality point for every network neighbor a vertex has, but not all neighbors are equivalent. Instead of awarding vertices just one point for each neighbor, eigenvector centrality gives each vertex a score proportional to the sum of the scores of its neighbors. Both global Eigenvector centrality ($E_g$) and local Eigenvector centrality ($E_l$) of the interdependent network are evaluated.
\subsection{Usefulness of Closeness centrality} Closeness centrality is a measure of the degree to which an individual is near all other individuals in a network. It is the inverse of the sum of the shortest distances between each node and every other node in the network.

\section{Interdependent centrality metrics}
To show the importance of designing new metrics of centrality for interdependent networks, we will modify the state-of-the-art metrics to count the differences between networks \textit{P} and \textit{C}, and repeat our resiliency evaluation analysis under the assumption that the modified centrality metrics are used to choose the important nodes of the cyber-physical network.

\section{Experiments} In the following, we present the simulation results for an interdependent cyber-physical system. It is assumed that the CPS includes $N=600$ nodes in total, 300 nodes in cyber network and 300 nodes in the physical network. An initial set of nodes in $C$ are randomly selected and are assumed to be failed at time $t=0$. We are interested in investigating the propagation of this failure over time. A time horizon of $T=200$ time slots is considered.

\textless DATA PLOTS GO HERE \textgreater





\section{Main Observations}
\textless OBSERVATIONS ON DATA PLOTS GO HERE \textgreater





\section{CONCLUSIONS}



\addtolength{\textheight}{-12cm}   % This command serves to balance the column lengths
                                  % on the last page of the document manually. It shortens
                                  % the textheight of the last page by a suitable amount.
                                  % This command does not take effect until the next page
                                  % so it should come on the page before the last. Make
                                  % sure that you do not shorten the textheight too much.

%%%%%%%%%%%%%%%%%%%%%%%%%%%%%%%%%%%%%%%%%%%%%%%%%%%%%%%%%%%%%%%%%%%%%%%%%%%%%%%%



%%%%%%%%%%%%%%%%%%%%%%%%%%%%%%%%%%%%%%%%%%%%%%%%%%%%%%%%%%%%%%%%%%%%%%%%%%%%%%%%



%%%%%%%%%%%%%%%%%%%%%%%%%%%%%%%%%%%%%%%%%%%%%%%%%%%%%%%%%%%%%%%%%%%%%%%%%%%%%%%%
%\section*{APPENDIX}

%Appendixes should appear before the acknowledgment.

%\section*{ACKNOWLEDGMENT}

%The preferred spelling of the word ÒacknowledgmentÓ in America is without an ÒeÓ after the ÒgÓ. Avoid the stilted expression, ÒOne of us (R. B. G.) thanks . . .Ó  Instead, try ÒR. B. G. thanksÓ. Put sponsor acknowledgments in the unnumbered footnote on the first page.



%%%%%%%%%%%%%%%%%%%%%%%%%%%%%%%%%%%%%%%%%%%%%%%%%%%%%%%%%%%%%%%%%%%%%%%%%%%%%%%%




\begin{thebibliography}{99}

\bibitem{c1} Z. Huang, C. Wang, M. Stojmenovic, A. Nayak, ``Characterization of Cascading Failures in Interdependent Cyber-Physical Systems," IEEE Transactions on Computers, vol. 64, pp. 2158--2168, Aug 2015.

\bibitem{c2} H.Khamfroush, N. Bartolini, T. La Porta, A. Swami, J. Dillman, ``On Propagation of Phenomena
in Interdependent Networks, " IEEE Transactions on Network Science and Engineering, vol. 3, pp. 225--239, Oct 2016.

\begin{comment}
\bibitem{c3} H. Poor, An Introduction to Signal Detection and Estimation.   New York: Springer-Verlag, 1985, ch. 4.
\bibitem{c4} B. Smith, ÒAn approach to graphs of linear forms (Unpublished work style),Ó unpublished.
\bibitem{c5} E. H. Miller, ÒA note on reflector arrays (Periodical styleÑAccepted for publication),Ó IEEE Trans. Antennas Propagat., to be publised.
\bibitem{c6} J. Wang, ÒFundamentals of erbium-doped fiber amplifiers arrays (Periodical styleÑSubmitted for publication),Ó IEEE J. Quantum Electron., submitted for publication.
\bibitem{c7} C. J. Kaufman, Rocky Mountain Research Lab., Boulder, CO, private communication, May 1995.
\bibitem{c8} Y. Yorozu, M. Hirano, K. Oka, and Y. Tagawa, ÒElectron spectroscopy studies on magneto-optical media and plastic substrate interfaces(Translation Journals style),Ó IEEE Transl. J. Magn.Jpn., vol. 2, Aug. 1987, pp. 740Ð741 [Dig. 9th Annu. Conf. Magnetics Japan, 1982, p. 301].
\bibitem{c9} M. Young, The Techincal Writers Handbook.  Mill Valley, CA: University Science, 1989.
\bibitem{c10} J. U. Duncombe, ÒInfrared navigationÑPart I: An assessment of feasibility (Periodical style),Ó IEEE Trans. Electron Devices, vol. ED-11, pp. 34Ð39, Jan. 1959.
\bibitem{c11} S. Chen, B. Mulgrew, and P. M. Grant, ÒA clustering technique for digital communications channel equalization using radial basis function networks,Ó IEEE Trans. Neural Networks, vol. 4, pp. 570Ð578, July 1993.
\bibitem{c12} R. W. Lucky, ÒAutomatic equalization for digital communication,Ó Bell Syst. Tech. J., vol. 44, no. 4, pp. 547Ð588, Apr. 1965.
\bibitem{c13} S. P. Bingulac, ÒOn the compatibility of adaptive controllers (Published Conference Proceedings style),Ó in Proc. 4th Annu. Allerton Conf. Circuits and Systems Theory, New York, 1994, pp. 8Ð16.
\bibitem{c14} G. R. Faulhaber, ÒDesign of service systems with priority reservation,Ó in Conf. Rec. 1995 IEEE Int. Conf. Communications, pp. 3Ð8.
\bibitem{c15} W. D. Doyle, ÒMagnetization reversal in films with biaxial anisotropy,Ó in 1987 Proc. INTERMAG Conf., pp. 2.2-1Ð2.2-6.
\bibitem{c16} G. W. Juette and L. E. Zeffanella, ÒRadio noise currents n short sections on bundle conductors (Presented Conference Paper style),Ó presented at the IEEE Summer power Meeting, Dallas, TX, June 22Ð27, 1990, Paper 90 SM 690-0 PWRS.
\bibitem{c17} J. G. Kreifeldt, ÒAn analysis of surface-detected EMG as an amplitude-modulated noise,Ó presented at the 1989 Int. Conf. Medicine and Biological Engineering, Chicago, IL.
\bibitem{c18} J. Williams, ÒNarrow-band analyzer (Thesis or Dissertation style),Ó Ph.D. dissertation, Dept. Elect. Eng., Harvard Univ., Cambridge, MA, 1993.
\bibitem{c19} N. Kawasaki, ÒParametric study of thermal and chemical nonequilibrium nozzle flow,Ó M.S. thesis, Dept. Electron. Eng., Osaka Univ., Osaka, Japan, 1993.
\bibitem{c20} J. P. Wilkinson, ÒNonlinear resonant circuit devices (Patent style),Ó U.S. Patent 3 624 12, July 16, 1990.
\end{comment}





\end{thebibliography}




\end{document}
